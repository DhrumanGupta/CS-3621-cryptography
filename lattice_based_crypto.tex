\documentclass[12pt]{article}

% Based on the handwritten lecture notes "Lattice Based Crypto.pdf". :contentReference[oaicite:0]{index=0}

\usepackage[margin=1in]{geometry}
\usepackage{amsmath,amssymb,amsthm,mathtools}
\usepackage{bbm}
\usepackage{hyperref}
\usepackage{enumitem}
\usepackage{algorithm}
\usepackage{algpseudocode}

\hypersetup{
  colorlinks=true,
  linkcolor=blue,
  urlcolor=blue,
  citecolor=blue
}

\setlength{\parindent}{0pt}

\newtheorem{definition}{Definition}
\newtheorem{proposition}{Proposition}
\newtheorem{theorem}{Theorem}
\newtheorem{lemma}{Lemma}
\newtheorem{corollary}{Corollary}

\newcommand{\Z}{\mathbb{Z}}
\newcommand{\R}{\mathbb{R}}
\newcommand{\Zn}{\mathbb{Z}^n}
\newcommand{\Zq}{\mathbb{Z}_q}
\newcommand{\Znq}{\mathbb{Z}_q^n}
\newcommand{\norm}[1]{\left\lVert #1 \right\rVert}
\newcommand{\ip}[2]{\left\langle #1 , #2 \right\rangle}
\newcommand{\vol}{\operatorname{vol}}

\title{Lattice-Based Cryptography}
\author{Dhruman Gupta}
\date{December 8, 2025}

\begin{document}

\maketitle

\tableofcontents

\newpage


%------------------------------------------------------------
\section{Lattices}
%------------------------------------------------------------

\subsection{Basic Definitions}

\begin{definition}[Lattice]
  Let $v_1,\ldots,v_n \in \R^m$ be linearly independent vectors. The lattice
  generated by these vectors is
  \[
    L = L(v_1,\ldots,v_n)
      = \left\{ \sum_{i=1}^n a_i v_i \;:\; a_i \in \Z \right\}.
  \]
  Any such set $\{v_1,\ldots,v_n\}$ that generates $L$ is called a \emph{basis}
  of $L$, and $n$ is the \emph{dimension} of the lattice, $\dim L = n$.
\end{definition}

\begin{definition}[Integer Lattice]
  A lattice $L \subseteq \R^m$ is an \emph{integer lattice} if all vectors in
  $L$ have integer coordinates. Equivalently, $L$ is a subgroup of $\Z^m$ (for
  some $m \ge 1$).
\end{definition}

Integer lattices are especially convenient for computation.

\subsection{Additive Subgroup Characterization}

\begin{definition}[Additive Subgroup]
  A subset $L \subseteq \R^m$ is an \emph{additive subgroup} if it is closed
  under vector addition and subtraction: $v,w \in L \Rightarrow v \pm w \in L$.
\end{definition}

\begin{definition}[Discrete Additive Subgroup]
  An additive subgroup $L \subseteq \R^m$ is \emph{discrete} if there exists
  $\varepsilon > 0$ such that for all distinct $v,w \in L$,
  \[
    \norm{v - w} \ge \varepsilon.
  \]
\end{definition}

\begin{proposition}
  A subset $L \subseteq \R^m$ is a lattice if and only if it is a discrete
  additive subgroup of $\R^m$.
\end{proposition}

\subsection{Change of Basis Matrices}

Let $v_1,\ldots,v_n$ and $w_1,\ldots,w_n$ be two bases of the same lattice
$L \subseteq \R^n$. Then, the change of basis matrix is an integer matrix $A \in \mathbb{Z}^{n \times n}$. More precisely:
\[
  (v_1,\ldots,v_n) = (w_1,\ldots,w_n) A.
\]
$A$ takes $w$ to $v$, and $A^{-1}$ takes $v$ to $w$ (the inverse exists as it takes a basis to a basis). In both cases, the coefficients of vectors must be integers, so $A$ and $A^{-1}$ are integer matrices. This implies that $det(A) = \pm 1$.

%------------------------------------------------------------
\section{Fundamental Domains and Determinant}
%------------------------------------------------------------

\subsection{Fundamental Domain}

\begin{definition}[Fundamental Domain]
  Let $v_1,\ldots,v_n$ be a basis of a lattice $L \subseteq \R^m$. The
  \emph{fundamental domain} corresponding to this basis is
  \[
    \mathcal{F}(v_1,\ldots,v_n) =
    \left\{ \sum_{i=1}^n t_i v_i \;:\; 0 \le t_i < 1 \right\}.
  \]
\end{definition}

\begin{proposition}
  Let $L \subseteq \R^m$ be an $n$-dimensional lattice. Every vector
  $x \in \R^m$ can be uniquely written as
  \[
    x = \ell + f,
  \]
  where $\ell \in L$ and $f \in \mathcal{F}(v_1,\ldots,v_n)$ for any basis
  $v_1,\ldots,v_n$ of $L$.
\end{proposition}

\begin{proof}
    Let $x \in \R^m$. We can write $x$ as a linear combination of the basis vectors with real coefficients:
    \[
        x = t_1 x_1 + \cdots + t_n x_n
    \]
    for some $t_i \in \R$. Define $a_i = \lfloor t_i \rfloor$. Since $a_i \in \Z$ and $0 \leq t_i - a_i < 1$, we have
    \[
        f = x - \sum_{i=1}^n a_i x_i \in \mathcal{F}(x_1,\ldots,x_n).
    \]
    Furthermore,
    \[
        \ell = \sum_{i=1}^n a_i x_i \in L.
    \]
    Therefore,
    \[
        x = f + \ell.
    \]
\end{proof}

\subsection{Determinant of a Lattice}

\begin{definition}[Determinant / Covolume]
  For an $n$-dimensional lattice $L \subseteq \R^m$, the \emph{determinant} of
  $L$, denoted $\det(L)$, is the $n$-dimensional volume of any fundamental
  domain $\mathcal{F}(v_1,\ldots,v_n)$. It is also called the
  \emph{covolume} of $L$.
\end{definition}

\begin{proposition}
    Let $V$ be the $m \times n$ matrix whose columns are the basis vectors
$v_1,\ldots,v_n$. When $m=n$ (i.e.\ $L\subseteq\R^n$ is full-dimensional), the
volume of the fundamental domain is
\end{proposition}

\begin{proof}
    The volume of the fundamental domain is given by
    \[
        \vol(\mathcal{F}) = \int_{\mathcal{F}} \, dx_1 \cdots dx_n.
    \]

    To compute this integral, make the change of variables
    \[
        (x_1, \dots, x_n) = t_1 v_1 + \cdots + t_n v_n,
    \]
    which can be written in matrix form as $x = V t$.\\

    Note that $\mathcal{F} = V C_n$, where $C_n = [0,1]^n$. By the change of variables formula,
    \[
        \vol(\mathcal{F}) = \int_{C_n} |\det(V)| \, dt_1 \cdots dt_n = |\det(V)|.
    \]
\end{proof}

\begin{corollary}
    For a fixed lattice $L$, every fundamental domain (for different bases)
    has the same volume $\det(L)$.
\end{corollary}

\[
  \vol\big(\mathcal{F}(v_1,\ldots,v_n)\big)
  = |\det(V)|.
\]

\begin{theorem}[Hadamard's Inequality]
  For any basis $v_1,\ldots,v_n$ of a lattice $L \subseteq \R^n$,
  \[
    \det(L) = \vol\big(\mathcal{F}\big)
    \le \prod_{i=1}^n \norm{v_i}.
  \]
  Equality holds if and only if the basis vectors are mutually orthogonal.
\end{theorem}


%------------------------------------------------------------
\section{Shortest and Closest Vector Problems}
%------------------------------------------------------------

\subsection{Exact Problems}

\begin{definition}[Shortest Vector Problem (SVP)]
  Given a lattice $L$, find a non-zero vector
  \[
    v_{\text{shortest}} \in L \setminus \{0\}
  \]
  of minimum Euclidean norm.
\end{definition}

\begin{definition}[Closest Vector Problem (CVP)]
  Given a lattice $L \subseteq \R^m$ and a target $w \in \R^m$, find
  \[
    v_{\text{closest}} = \arg\min_{v \in L} \norm{w - v}.
  \]
\end{definition}

The solutions need not be unique. CVP is known to be NP-hard, and SVP is
NP-hard under randomized reductions (under standard assumptions).

\subsection{Other Related Problems}

\begin{definition}[Shortest Basis Problem (SBP)]
  Given a lattice $L$, find a basis $v_1,\ldots,v_n$ that is ``short'' in some
  measure, e.g.\ minimizing $\max_i \norm{v_i}$ or $\sum_i \norm{v_i}^2$.
\end{definition}

\begin{definition}[Approximate SVP]
  For an approximation factor $\gamma(n) \ge 1$, the $\gamma$-approximate SVP
  problem asks: given a lattice $L$ of dimension $n$, find a non-zero
  $v \in L$ such that
  \[
    \norm{v} \le \gamma(n) \cdot \norm{v_{\text{shortest}}}.
  \]
\end{definition}

\begin{definition}[Approximate CVP]
  For an approximation factor $\gamma(n)\ge1$, given $L$ and $w$, find
  $v\in L$ such that
  \[
    \norm{w-v} \le \gamma(n)\cdot
    \norm{w - v_{\text{closest}}}.
  \]
\end{definition}

\subsection{Hermite's Theorem and Hermite Constant}

\begin{definition}[Hermite Constant]
  For dimension $n$, the \emph{Hermite constant} $\gamma_n$ is defined so that
  every lattice $L$ of dimension $n$ contains a non-zero vector $v$ with
  \[
    \norm{v}^2 \le \gamma_n \cdot \det(L)^{2/n}.
  \]
\end{definition}

\begin{theorem}[Hermite's Theorem]
  For every $n$-dimensional lattice $L$ there exists a non-zero vector $v\in L$
  such that
  \[
    \norm{v} \le \sqrt{\gamma_n} \,\det(L)^{1/n}.
  \]
\end{theorem}

For large $n$, asymptotically
\[
  \frac{n}{2\pi e} \lesssim \gamma_n \lesssim \frac{n}{\pi e}.
\]

\subsection{Hadamard Ratio}

\begin{definition}[Hadamard Ratio]
  For a basis $B = (v_1,\ldots,v_n)$ of a lattice $L$,
  \[
    H(B) =
    \left(
      \frac{\det(L)}{\prod_{i=1}^n \norm{v_i}}
    \right)^{1/n}.
  \]
  We have $0 < H(B) \le 1$ from the Hadamard inequality. The more orthogonal the basis $B$ is, the closer
  $H(B)$ is to $1$.
\end{definition}

\subsection{Minkowski's Theorem}

\begin{theorem}[Minkowski]
  Let $L \subseteq \R^n$ be an $n$-dimensional lattice and let $S \subseteq
  \R^n$ be a symmetric convex set (i.e.\ $x \in S \Rightarrow -x \in S$) such
  that
  \[
    \vol(S) > 2^n \det(L).
  \]
  Then $S$ contains a non-zero lattice vector, i.e.\ there exists
  $v \in L \setminus \{0\}$ with $v \in S$. If $S$ is closed, the strict
  inequality can be replaced by $\ge$.
\end{theorem}

\begin{proof}
    Let $S$ be a symmetric convex set such that $\vol(S) > 2^n \det(L)$. Consider the set $\frac{1}{2}S$.
    \[
        \vol\left(\frac{1}{2}S\right) = \left(\frac{1}{2}\right)^n \vol(S) > 2^n \det(L) \cdot \left(\frac{1}{2}\right)^n = \det(L).
    \]

    Every vector $a \in \R^m$ can be uniquely written as $a = v + w$, where $w \in L$ and $v \in \mathcal{F}$ (the fundamental domain).\\

    Define the map $f: \frac{1}{2}S \to \mathcal{F}$ by $f\left(\frac{1}{2}x\right) = w$, where $w$ is the non-lattice part of $\frac{1}{2}x$. This map consists of finitely many translations by lattice vectors, and since $S$ is bounded, $f$ preserves volume.\\

    Since $\vol(\frac{1}{2}S) > \vol(\mathcal{F})$, the map $f$ cannot be injective. Therefore, there exist distinct points $\frac{1}{2}x, \frac{1}{2}y \in \frac{1}{2}S$ such that $f(\frac{1}{2}x) = f(\frac{1}{2}y) = w$.\\

    Write $\frac{1}{2}x = v_1 + w$ and $\frac{1}{2}y = v_2 + w$ with $v_1 \neq v_2$. Then
    \[
        \frac{1}{2}(x - y) = v_1 - v_2 \neq 0.
    \]
    Since $S$ is convex and symmetric, $\frac{1}{2}(x - y) \in S$. Thus $S$ contains a non-zero lattice vector.
\end{proof}

By applying Minkowski's theorem to an $n$-dimensional hypercube of side
length $2\det(L)$, one obtains Hermite's bound on the length of the shortest vector.

\subsection{Gaussian Heuristic}

Instead of a hypercube, one can consider an $n$-dimensional ball of radius $r$
and approximate its volume. The \emph{Gaussian heuristic} predicts that for a
``random'' $n$-dimensional lattice $L$,
\[
  \lambda_1(L) \approx \sigma(L)
  := \sqrt{\frac{n}{2\pi e}} \,\det(L)^{1/n},
\]
meaning the shortest vector length is close (up to small constant factors) to
this value, with high probability.

%------------------------------------------------------------
\section{Babai's Algorithm}
%------------------------------------------------------------

\subsection{Orthogonal Bases}

If a basis of $L$ is orthogonal, many lattice problems are easy.

Let $v_1,\ldots,v_n$ be an orthogonal basis of $L$. Any $v\in L$ can be
written uniquely as
\[
  v = \sum_{i=1}^n a_i v_i, \quad a_i \in \Z.
\]

\paragraph{SVP with orthogonal basis.} The shortest non-zero vector is simply
the basis vector $v_i$ of minimum length.

\paragraph{CVP with orthogonal basis.} Given $w\in\R^n$, write
\[
  w = \sum_{i=1}^n t_i v_i, \quad t_i \in \R.
\]
Then the closest lattice vector is
\[
  v = \sum_{i=1}^n a_i v_i, \quad
  a_i = \operatorname{round}(t_i).
\]

\subsection{Babai's Algorithm for Nearly Orthogonal Bases}

For a basis that is \emph{nearly} orthogonal (e.g.\ after lattice reduction),
\emph{Babai's rounding algorithm} gives an approximate solution to CVP.

\begin{algorithm}
    \caption{Babai's Nearest-Plane Algorithm}
    \begin{algorithmic}[1]
    \Require Basis $v_1,\ldots,v_n$ of a lattice $L$; target vector $w \in \mathbb{R}^n$
    \Ensure Approximate closest lattice vector $v \in L$
    \State Compute $t_1,\ldots,t_n \in \mathbb{R}$ such that 
           $w = \sum_{i=1}^n t_i v_i$
    \For{$i = 1$ to $n$}
        \State $a_i \gets \operatorname{round}(t_i)$
    \EndFor
    \State \Return $v = \sum_{i=1}^n a_i v_i$
    \end{algorithmic}
\end{algorithm}
    
    
When the basis is sufficiently close to orthogonal, the returned $v$ is often
the closest lattice vector, or at least within a small approximation factor.\\

Analysis for how close $v$ is to the actual vector is omitted.

%------------------------------------------------------------
\section{Learning With Errors (LWE) and Hard Lattice Problems}
%------------------------------------------------------------

\subsection{The LWE Problem}

Fix positive integers $n$ (dimension) and $q \ge 2$ (modulus), and let $\chi$
be a probability distribution over $\Z_q$ (the ``error distribution``).\\

Let $s \in \Z_q^n$ be a secret vector. Define the distribution $A_{s,\chi}$
over $\Z_q^n \times \Z_q$ by sampling
\[
  a \leftarrow \Z_q^n \text{ uniformly}, \qquad
  e \leftarrow \chi, \qquad
  b = \ip{a}{s} + e \pmod{q},
\]
and outputting $(a,b)$.

\begin{definition}[LWE with parameters $(n,q,\chi)$]
An algorithm solves LWE if, given arbitrarily many independent samples from
$A_{s,\chi}$ (for a fixed but unknown $s$), it can recover $s$ with
non-negligible probability.
\end{definition}

\subsection{Some hardness results, and worst-case lattice problems}

From here on, $\lambda_1(L)$ denotes the length of the shortest non-zero vector in $L$.

\begin{definition}[GapSVP]
  Let $L$ be a lattice. For a function $\beta(n)$ (the gap), the problem
  $\textsf{GapSVP}_{\beta}$ asks, given $L$, to decide whether
  $\lambda_1(L) \le 1$ or $\lambda_1(L) > \beta(n)$, under the promise that one
  of these is the case.
\end{definition}

LWE is believed to be at least as hard as approximating $\lambda_1(L)$ within
polynomial factors for worst-case lattices (i.e.\ certain GapSVP and related
problems), via classical and quantum reductions.

\subsection{Small Integer Solution (SIS) Problem}

SIS is often described as a ``dual'' of LWE.

\begin{definition}[SIS]
  Let $A \in \Z_q^{n \times m}$ be a matrix whose columns
  $a_1,\ldots,a_m \in \Z_q^n$ are chosen uniformly at random. The SIS problem
  with parameter $B$ asks for a non-zero integer vector
  $b = (b_1,\ldots,b_m) \in \Z^m$ such that
  \[
    A b \equiv 0 \pmod{q} \quad\text{and}\quad \norm{b} \le B.
  \]
\end{definition}

For appropriate ranges of $q$ and $B$, solving SIS on average is as hard as solving certain worst-case lattice problems such as GapSVP and SIVP.

\subsection{Shortest Independent Vectors Problem (SIVP)}

\begin{definition}[SIVP]
  Given an $n$-dimensional lattice $L$, find a set of $n$ linearly independent
  lattice vectors $v_1,\ldots,v_n$ such that the maximum length
  $\max_i \norm{v_i}$ is as small as possible among all bases of $L$.
\end{definition}


\subsection{Bounded Distance Decoding (BDD)}

\begin{definition}[BDD]
  Let $L$ be a lattice and $d>0$. In the \emph{bounded distance decoding} (BDD)
  problem, one is given a target $x$ that is promised to be within distance
  $d$ of the lattice:
  \[
    \exists\,v \in L: \norm{x-v} \le d.
  \]
  The goal is to find this (or some) closest lattice vector $v$.
\end{definition}

If $d < \lambda_1(L)/2$, then the closest vector is unique. BDD can be seen as a promise version of CVP in which the target is guaranteed to be close to the lattice.

\subsection{Dual Lattice}

\begin{definition}[Dual Lattice]
  Let $L \subseteq \R^n$ be a lattice. The \emph{dual lattice} $L^{\ast}$ is
  defined as
  \[
    L^{\ast} = \{\,y \in \R^n : \ip{y}{x} \in \Z \text{ for all } x \in L\,\}.
  \]
\end{definition}

For example, for $t>0$, the dual of $t\Z^n$ is $(1/t)\Z^n$.

\subsection{Discrete Gaussian and SIVP}

\begin{definition}[Discrete Gaussian Distribution]
  Let $L$ be a lattice and $r>0$. The \emph{discrete Gaussian} over $L$ with
  parameter $r$ assigns probability proportional to
  \[
    \exp\!\left(-\pi \frac{\norm{x}^2}{r^2}\right)
  \]
  to each $x \in L$ (properly normalized).
\end{definition}

Typical samples from this distribution have norm around $r\sqrt{n}$ and are
substantially longer than the shortest vector.

\subsection{Reductions Relating LWE and Lattice Problems}

Very roughly:
\begin{itemize}[leftmargin=*]
  \item GapSVP and SIVP reduce to BDD on certain lattices.
  \item BDD can in turn be reduced to LWE: given access to an LWE oracle, one
    can solve BDD within a radius on the order of
    $\lambda_1(L)/\text{poly}(n)$.
  \item Therefore, efficiently solving LWE (for typical parameters) would allow
    one to solve worst-case lattice problems like GapSVP and SIVP in
    (sub-)exponential regimes where no such algorithms are known.
\end{itemize}

%------------------------------------------------------------
\section{An LWE-Based Public-Key Encryption Scheme}
%------------------------------------------------------------

We now describe a (simplified) public-key encryption scheme based on LWE.

\subsection{Parameters}

These parameters are chosen so that decryption is possible, while maintaining good security.

\begin{itemize}[leftmargin=*]
  \item Dimension $n$.
  \item Prime modulus $q \in [n^2, 2n^2]$.
  \item Error distribution $\chi$ discrete normal distribution with variance $\alpha q$.
  \item $\alpha = \frac{1}{\sqrt(n)\log^2 n}$
  \item $m = 1.1n\log q$
\end{itemize}

\subsection{Key Generation}

\begin{itemize}[leftmargin=*]
  \item Sample secret key
  \[
    s \leftarrow \Z_q^n.
  \]
  \item Sample $m$ (polynomial in $n$) LWE samples
  \[
    (a_i,b_i) \leftarrow A_{s,\chi}, \quad i=1,\ldots,m,
  \]
  i.e.\ $a_i \leftarrow \Z_q^n$, $e_i \leftarrow \chi$ and
  $b_i = \ip{a_i}{s} + e_i \pmod{q}$.
  \item Secret key:
  \[
    \textsf{sk} = s.
  \]
  \item Public key:
  \[
    \textsf{pk} = \{(a_i,b_i)\}_{i=1}^m.
  \]
\end{itemize}

\subsection{Encryption}

To encrypt a bit $x \in \{0,1\}$ using the public key $\{(a_i,b_i)\}$:
\begin{itemize}[leftmargin=*]
  \item Choose a random subset $S \subseteq \{1,\ldots,m\}$.
  \item Compute
  \[
    a = \sum_{i \in S} a_i \pmod{q}, \qquad
    b' = \sum_{i \in S} b_i \pmod{q}.
  \]
  \item If $x=1$, also add $\lfloor q/2 \rfloor$ to $b'$:
  \[
    b =
    \begin{cases}
      b' & \text{if } x = 0,\\[2mm]
      b' + \lfloor q/2 \rfloor \pmod{q} & \text{if } x = 1.
    \end{cases}
  \]
  \item The ciphertext is $(a,b)$.
\end{itemize}

\subsection{Decryption}

Given ciphertext $(a,b)$ and secret key $s$:
\begin{itemize}[leftmargin=*]
  \item Compute
  \[
    t = b - \ip{a}{s} \pmod{q}.
  \]
  \item Interpret $t$ as an integer in $(-q/2,q/2)$ and output
  \[
    \hat{x} =
    \begin{cases}
      0 & \text{if } t \text{ is closer to } 0 \text{ than to } \lfloor q/2 \rfloor,\\
      1 & \text{otherwise}.
    \end{cases}
  \]
\end{itemize}

\subsection{Correctness}

Each LWE sample satisfies
\[
  b_i = \ip{a_i}{s} + e_i \pmod{q},
\]
so for the chosen subset $S$ we have
\[
  b = \sum_{i \in S} b_i
    = \ip{\sum_{i \in S} a_i}{s}
      + \sum_{i \in S} e_i
     = \ip{a}{s} + E \pmod{q},
\]
where $E = \sum_{i \in S} e_i$ is the accumulated error. If $x=0$ we send $b' = b$, and if $x=1$ we send $b' = b + \lfloor q/2 \rfloor$.\\

On decryption,
\[
  t =
  c' - \ip{a}{s} =
  \begin{cases}
    E & \text{if } b=0,\\[1mm]
    E + \lfloor q/2 \rfloor & \text{if } b=1,
  \end{cases}
  \pmod{q}.
\]

So, we only have an error if $|E| \ge q/4$. Let's calculate the probability of this.\\

We sum of $\leq m$ terms, each with std $\alpha q$, so std of $E \leq \sqrt(m) \alpha q < \frac{q}{\log n}$. $\chi$ is normal, so by Chernoff's bound:
\begin{align*}
    P(|E| &\ge \frac{q}{4}) \leq 2e\exp(-\frac{q^2 \log^2 n}{4^2 2 q^2})\\
    &= 2n^{-\frac{\log n}{32}}
\end{align*}

This is negligible, so decryption succeeds except with negligible probability.

\subsection{Security}

Let $\mathcal{U}$ be the uniform distribution over $\mathcal{Z}_q^n \times \mathcal{Z}_q$. I will state and prove the following lemmas, and then argue security.

\begin{lemma}
    Let $n \geq 1$, $2 \leq q \leq poly(n)$ a prime, and $\chi$ some distribution over $\mathcal{Z}_q$. Assume there exists a PPT adversary $\mathcal{A}$ that accepts w.p $1-\epsilon_1(n)$ when $x \in A_{s,\chi}$, and rejects w.p $1 - \epsilon_2(n)$ when $x \in \mathcal{U}$, where $\epsilon_1(n), \epsilon_2(n) \in 2^{-\Omega(n)}$. Then, we can construct a PPT adversary $\mathcal{B}$ that can output $s$ given samples from $A_{s,\chi}$ with non negligible probability.
\end{lemma}

\begin{proof}
    Let $s = (s_1, \ldots, s_n)$. I will show how to find $s_1$. Repeat the process for the rest.

    For all $k \in \mathbb{Z}_q$, consider the transformation $T_k: \mathbb{Z}_q^n \times \mathbb{Z}_q \to \mathbb{Z}_q^n \times \mathbb{Z}_q$ defined by:
    \[
        T_k(a,b) = (a + (r, 0, \cdots, 0), b + rk)
    \]
    where $r$ is uniformly random in $\mathbb{Z}_q$.\\

    Now, note that if $k = s_1$, then $T_k(A_{s,\chi}) = A_{s,\chi}$. If $k \neq s_1$, then $T_k(A_{s,\chi}) = \mathcal{U}$.\\

    Use $A$ to see whether $T_k(A_{s,\chi}) = A_{s,\chi}$. If so, then accept. Reject otherwise. There are only $poly(n)$ such $k$ to try, so we can try them all.
\end{proof}

\begin{lemma}
Let $n, q \geq 1$, $\chi$ some distribution over $\mathbb{Z}_q$. Assume there exists a PPT adversary $\mathcal{A}$ that can distinguish non-negligible samples of $A_{s,\chi}$ from $\mathcal{U}$. Then, we can construct a PPT adversary $\mathcal{B}$ that for all $s$, accepts $x \in A_{s,\chi}$ w.p exponentially close to $1$, and rejects $x \in \mathcal{U}$ w.p exponentially close to $1$.
\end{lemma}

\begin{proof}
    Let $t \in \mathbb{Z}_q^n$. Consider the transformation $f_t: \mathbb{Z}_q^n \times \mathbb{Z}_q \to \mathbb{Z}_q^n \times \mathbb{Z}_q$ defined by:
    \[
        f_t(a,b) = (a, b + \ip{t}{a})
    \]
    Note that $f_t(A_{s,\chi}) = A_{s+t,\chi}$, and $f_t(\mathcal{U}) = \mathcal{U}$.\\

    Choose a $t$ uniformly. Estimate the probability of acceptance of $\mathcal{A}$ on $f_t(A_{s,\chi})$ and $f_t(\mathcal{U})$ to degree $\pm \frac{1}{poly(n)}$.\\

    If these differ, then it means that $f_t(\text{input}) = A_{s+t,\chi}$. So accept. Otherwise reject.\\

    Now, repeat polynomially many times. If the input is indeed $A_{s,\chi}$, then we will get to some $A_{s+t,\chi}$ s.t $\mathcal{A}$ will distinguish between that and $\mathcal{U}$. So if we do not find that in polynomial time, then with probability exponentially close to $1$, the input is not $A_{s,\chi}$.
\end{proof}

\subsubsection{Proof that this is IndCPA}

We are working under the assumption that with these parameters, the LWE problem is hard. We will show that this scheme is IndCPA secure under this assumption.\\

In the IndCPA game, the player sends two messages $m_0, m_1$ to the challenger, and then tries to distinguish between which one was encrypted. Since the messages in this setting are a bit, the game boils down to:
\begin{enumerate}
    \item Challenger is given the public key.
    \item $b \in \{0,1\}$ is chosen uniformly, and encrypted to get $c$.
    \item Challenger has to guess whether $b = 0$ or $b = 1$.
\end{enumerate}

Say $\mathcal{A}$ is a PPT adversary that wins the game with probability $\frac{1}{2} + \epsilon(n)$ where $\epsilon$ is non-negligible.\\

Generate ${(a_i, b_i)}_{i=1}^m$ uniformly at random, instead of the LWE distribution. Now, for any subset $S$ of this, the pair $(\sum_{i \in S} a_i, \sum_{i \in S} b_i)$ is almost uniformly random. The encryption of $0$ and $1$ are theoretically indistinguishable, so no adversary can distinguish them.\\

As $\mathcal{A}$ cannot distinguish between the two, we can use it to distinguish between the LWE distribution and the uniform distribution. Lemma 1 and 2 then imply that this adversary can break LWE.\\

This is contradicting our assumption that the LWE problem is hard. So, the scheme is IndCPA secure.


%------------------------------------------------------------
\section*{Summary}

These notes have introduced:
\begin{itemize}[leftmargin=*]
  \item Two early cryptosystems (a toy scheme and Merkle's subset-sum scheme)
        that can be attacked via finding short vectors in lattices.
  \item Basic lattice theory, including bases, fundamental domains, and the
        determinant.
  \item Central computational problems (SVP, CVP, SIS, SIVP, BDD) and
        geometric tools (Minkowski, Hermite, Gaussian heuristic, Babai's
        algorithm).
  \item The Learning With Errors problem and its connection to worst-case
        lattice problems.
  \item A concrete LWE-based public-key encryption scheme, together with
        correctness and a sketch of its security based on LWE hardness.
\end{itemize}

\end{document}
